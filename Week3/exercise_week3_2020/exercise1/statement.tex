% Options for packages loaded elsewhere
\PassOptionsToPackage{unicode}{hyperref}
\PassOptionsToPackage{hyphens}{url}
%
\documentclass[
]{article}
\usepackage{lmodern}
\usepackage{amssymb,amsmath}
\usepackage{ifxetex,ifluatex}
\ifnum 0\ifxetex 1\fi\ifluatex 1\fi=0 % if pdftex
  \usepackage[T1]{fontenc}
  \usepackage[utf8]{inputenc}
  \usepackage{textcomp} % provide euro and other symbols
\else % if luatex or xetex
  \usepackage{unicode-math}
  \defaultfontfeatures{Scale=MatchLowercase}
  \defaultfontfeatures[\rmfamily]{Ligatures=TeX,Scale=1}
\fi
% Use upquote if available, for straight quotes in verbatim environments
\IfFileExists{upquote.sty}{\usepackage{upquote}}{}
\IfFileExists{microtype.sty}{% use microtype if available
  \usepackage[]{microtype}
  \UseMicrotypeSet[protrusion]{basicmath} % disable protrusion for tt fonts
}{}
\makeatletter
\@ifundefined{KOMAClassName}{% if non-KOMA class
  \IfFileExists{parskip.sty}{%
    \usepackage{parskip}
  }{% else
    \setlength{\parindent}{0pt}
    \setlength{\parskip}{6pt plus 2pt minus 1pt}}
}{% if KOMA class
  \KOMAoptions{parskip=half}}
\makeatother
\usepackage{xcolor}
\IfFileExists{xurl.sty}{\usepackage{xurl}}{} % add URL line breaks if available
\IfFileExists{bookmark.sty}{\usepackage{bookmark}}{\usepackage{hyperref}}
\hypersetup{
  pdftitle={Markov chain sampling},
  hidelinks,
  pdfcreator={LaTeX via pandoc}}
\urlstyle{same} % disable monospaced font for URLs
\usepackage[margin=1in]{geometry}
\usepackage{graphicx,grffile}
\makeatletter
\def\maxwidth{\ifdim\Gin@nat@width>\linewidth\linewidth\else\Gin@nat@width\fi}
\def\maxheight{\ifdim\Gin@nat@height>\textheight\textheight\else\Gin@nat@height\fi}
\makeatother
% Scale images if necessary, so that they will not overflow the page
% margins by default, and it is still possible to overwrite the defaults
% using explicit options in \includegraphics[width, height, ...]{}
\setkeys{Gin}{width=\maxwidth,height=\maxheight,keepaspectratio}
% Set default figure placement to htbp
\makeatletter
\def\fps@figure{htbp}
\makeatother
\setlength{\emergencystretch}{3em} % prevent overfull lines
\providecommand{\tightlist}{%
  \setlength{\itemsep}{0pt}\setlength{\parskip}{0pt}}
\setcounter{secnumdepth}{-\maxdimen} % remove section numbering

\title{Markov chain sampling}
\usepackage{etoolbox}
\makeatletter
\providecommand{\subtitle}[1]{% add subtitle to \maketitle
  \apptocmd{\@title}{\par {\large #1 \par}}{}{}
}
\makeatother
\subtitle{Week3-ex1, problem statement}
\author{}
\date{\vspace{-2.5em}}

\begin{document}
\maketitle

The purpose of this exercise is to study the properties of Markov chains
and how they can be used to produce samples for Monte Carlo estimation.

Consider a Markov chain defined as follows:

\begin{itemize}
\item set $\theta^{(0)} = C$, where $C$ is some constant number.
\item for $i=1,\dots$ sample $\theta^{(i)} \sim N(\phi \theta^{(i-1)},\sigma^2)$ where $\phi\in[0,1)$ is a parameter controlling the autocorrelation between samples.
\end{itemize}

It can be shown that as \(i \rightarrow \infty\) (that is, at the limit
of long chain) the \emph{marginal} distribution of each \(\theta^{(i)}\)
is a Gaussian with mean zero and variance
\(\text{Var}[\theta^{(i)}] = \frac{\sigma^2}{1-\phi^2}\) and that the
correlation between \(\theta^{(i)}\) and \(\theta^{(i+t)}\) is
\(\text{Corr}([\theta^{(i)},\theta^{(i+t)}] = \phi^t\)

\begin{enumerate}
\item What are the variance of $\theta^{(i)}$ and the correlation between $\theta^{(i)}$ and $\theta^{(i+t)}$ at the limit of large $i$ when $\phi \rightarrow 0$ and when $\phi \rightarrow 1$. Assume that $\sigma$ is fixed in these cases.
\item Fill in the below table\\ \\
\begin{tabular}{c|c|c|c}
$\text{Var}[\theta^{(i)}]$ & $\phi$ & $\sigma^2$ & $\text{Corr}[\theta^{(i)},\theta^{(i+1)}]$\\
\hline
1 &     &  1   &  \\ 
1 & 0.5 &      &   \\
1 &     &  0.2 & \\
1 & 0.1 &   & \\
\end{tabular}
%What are the values of $\sigma^2$ so that the marginal variance of $\theta^{(i)$ is one 
\item Implement the above Markov chain with R and use it to sample random realizations of $\theta^{(i)}$ where $i=1,\dots,100$ with the parameter values given in the above table. As an initial value use $C=10$. Plot the sample chain and based on the visual inspection, what can you say about the convergence and mixing properties of the chain with the different choices of $\phi$?
\item Choose the parameter combination where $\sigma^2=0.2$ from the above table. Run three Markov chains with initial values $C_1 = 10$, $C_2=-10$ and $C_3=5$. Find a burn-in value at which the chains have converged according to the PSRF ($\hat{R}$) statistics. This is implemented in function \texttt{Rhat} in RStan. Note, $m=100$ samples might not be enough here.
\end{enumerate}

Note! This is a Markov chain that is constructed very differently from
how Stan constructs the Markov chains to sample from the posterior
distributions. However, the properties related to autocorrelation and
initial value are analogous.

\hypertarget{grading}{%
\section{Grading:}\label{grading}}

\textbf{Total 10 points:} 2 points for correct answer for step 1. 3
points for correct answer to step 2. 3 points for correct answer for
step 3. 2 points for correct answer for step 4. \textbf{Note}, You
should not penalize from wrong parameter values in step 3 and 4 if the
table was filled wrong in step 2.

\end{document}
